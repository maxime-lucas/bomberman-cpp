\documentclass[a4paper]{report}
\usepackage[utf8]{inputenc}
\usepackage[francais]{babel} 
\usepackage{datetime}
\usepackage{geometry}
\usepackage{listingsutf8}
\usepackage{xcolor}
\usepackage{graphicx}
\geometry{hmargin=1.5cm,vmargin=1.7cm}
\renewcommand{\thesection}{\Roman{section}}
\renewcommand{\contentsname}{Table des matières}
\lstset{%
	float=hbp,basicstyle=\footnotesize\ttfamily\color{black},%
	columns=fixed,tabsize=4,frame=single,%
	showspaces=false,showstringspaces=false,numbers=left,%
	numberstyle=\tiny\ttfamily\color{black},%
	breaklines=true, breakindent=3em, breakautoindent=true,%
	captionpos=t,xleftmargin=-1em,xrightmargin=-1em,lineskip=0pt,%
	numbersep=1em,backgroundcolor=\color{white},%
	keywordstyle=\bfseries\color{blue},%
	literate=%
         {à}{{\`a}}1
         {í}{{\'i}}1
         {î}{{\^i}}1
         {é}{{\'e}}1
         {è}{{\`e}}1
         {ê}{{\^e}}1
}

\begin{document}

\begin{titlepage}
	\begin{center}
		\vspace*{10cm}
	
		\textbf{\Huge Réalisation d'un Bomberman}
		
		\vspace{0.5cm}
		\LARGE{Projet Tutoré de Programmation Orientée Objet}
		
		\vspace{1.5cm}
		\textbf{Kevin CHIDIAC et Maxime LUCAS}
		
		\vfill
		
		\today	
	\end{center}
\end{titlepage}

\tableofcontents

\newpage


\begin{center}
\section{Préface}
\end{center}

\subsection{Introduction}
Hello World !\\
Ne vous êtes-vous jamais demandé comment les jeux étaient développés ?\\
Lorsque vous étiez petit, n’avez-vous jamais aspiré à réaliser un jeu vidéo ?\\
Pourtant, c’est bien ce rêve enfantin que nous avons réalisé pendant 4 semaines.\\
Bienvenue dans notre rapport de cette expérience des plus formatrices.\\
Cet ouvrage a pour but de vous expliquer le raisonnement et la démarche de réalisation que nous avons empruntée pour aboutir à un tel résultat.

\subsection{Problématique}
Tout d’abord, la problématique originale est la suivante :\\
Il nous faut développer une esquisse du célèbre jeu Bomberman, dans le langage orienté
objet C++. Ainsi, nous devons fédérer un ensemble de classes, répondant au mieux aux
exigences d’un tel sujet, et respectant évidemment les principes mêmes de la programmation orientée objet, à savoir : l’encapsulation, l’héritage, la composition, le
polymorphisme, etc...\\
De plus, il est conseillé d’utiliser la bibliothèque graphique Allegro, disposant d’une
documentation conséquente, bien que vieillissante.\\
Enfin, concernant les limites du sujet, pour rentrer dans le cahier des charges, il nous
faut gérer l’affichage du labyrinthe, le déplacement du personnage à travers ce dernier,
détecter les collisions du protagoniste avec les blocs environnants, et en induire l’impact sur son déplacement.\\

\end{document}
